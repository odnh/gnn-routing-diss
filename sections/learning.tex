\chapter{Reinforcement learning methodology}
\label{chapter:learning}
\todo[inline]{think of renaming this chapter to be less generic, more focussed on my project}

- the point of this chapter is to:
  - describe how the different policies have been designed (defo GNN, maybe LSTM, hopefully iter)
  - have some nice diagrams of the policy design maybe?
  - talk about modifications/translations of observation and action spaces to make them usable
  - talk about GNNs (encode -> process -> decode) and why (can vary obs + action size!)
  
 \section{Introduction}
 \todo{give an intro to contents of chapter, why this chapter exists and what it seeks to do}
 

\section{Environment interface}
\todo{think all this stuff is in the previous section, maybe think about removing or putting some of that here... Need to mention that gamma is a learned param somewhere though. Is also vvv important to mention how iterative version works}

\section{Standard policies}
\todo{decribe MLP \& LSTM baselines. metion the thinking behind them. say why they cannot generalise}

\section{GNN policy}
\todo{start with summary para of the section. then detail on how GNNs work (that isn't already in background). talk about the encode-process-decode model how it is the most general etc therefore we really are using deep learning as not manually crafting how stuff connects at all. Use a diagram for this. Talk about how the inputs are set etc (try not to be too engineeringy though)}

\section{Iterative GNN policy}
\todo{summary para. mention placeto model. explain the embedding strategy. explain whichever reward idea I end up using. again, diagrams}
