\chapter{Evaluation}

\subsection{Aims}

- in which I discuss the goals of the evaluation (what does it seek to show?). This should roughly map on to contributions given in earlier chapters
- talk about using progressive randomisation

\subsection{Experimental setup}

- talk about the training setup, hyperparameter tuning, the aget used (PPO2) and CITE
- talk about how dms are generated and why this is ok

\subsection{Baselines}

- mlp \& lstm
- softmin and random
- maybe also Raeke (if I get working)

\subsection{Learning static routing}

- talk about single sequence, single cycle (all should work)

\subsection{Learning cyclic routing}

- talk about adding cycles and training on one / multiple seqs. Still only test from training set though

\subsection{Generalising to unseen sequences}

- talk about changing just the dms, changing the cycle length, even changing dm type

\subsection{Generalising to unseen graphs}

- talk about why LSTM and MLP just can't do this
- talk about why is important (e.g. drop node, drop link, add node, add link) (amybe do this earlier in diss)
- compare against MLP where modify the routing using some heuristic maybe???
- do drop/add node/link
- do generate graph from same distribution
- do out of distribution (this will probably really fail...

\subsection{Real world}

- test on TOTEM/Abilene with cross-fold validation