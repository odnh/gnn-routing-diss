\chapter{Conclusions and future work}
\label{chapter:conclusions}

\section{Conclusions}
In this work we have sought to provide a generalisable approach to data-driven using deep RL with GNNs. This has consisted of providing a detailed specification of the problem and presenting an environment which can be used to experiment with different techniques to see how well they perform. We also took the approach introduced by Valadarsky et al. and modified it extensively so that it both worked to spread traffic across multipath routes and so that it could work with GNNs. In addition, we designed GNN policy architectures that could be used in conjunction with this problem and more generally provided a connection between an RL library and variable-sized graph network models.

After presenting previous work as well as our new additions we performed extensive experiments on the different policy architectures in different scenarios to assess how well they generalise to different demand matrices followed by different kinds of regularity in demand sequences before finally looking at generalisation to different network topologies. These experiments showed that all these policies can learn to efficiently route a demand matrix, in fact multiple. They also showed that GNNs are able to generalise learned routings. Unfortunately they showed that we were unable to make any of the policies learn to route demand matrices based on a history of previous demands beyond providing a good oblivious routing that did a good job of minimising the utility function (in this case maximum link utilisation).

Overall there has been the presentation of new techniques and a new library along with a mixture of successes and failures. Fortunately we can draw on these failures to inspire new work.


\section{Future work}
As previous work as suggested that the approaches taken in this paper can be successful, we feel it is important for these approaches to be explored further. On core part of the approach taken has been in designing a good way to reduce the size of the action space. An important part of this was in designing a mapping from edge weights to a routing strategy. We feel that an exploration of different techniques mapping edge weights or indeed any intermediate structure to a routing could provide interesting results. Furthermore, the way that demand information is input to the network could be greatly improved through numerous techniques and possibly hierarchical RL would be an interesting area to explore here.

Beyond mappings of observations and outputs there are many different ways in which the learning itself could be modified. These include using different learning algorithms to PPO and employing different techniques for graph generalisation such as the iterative approach. Another aspect that could be looked into here are modifications to the reward function with different properties that could aid exploration. Finally, we saw some successes with varied sequence length from the LSTM approach so if this could be combined with a method allowing for generalisation to different graphs it could lead to an approach achieving the best of both worlds.

Assuming the success of techniques presented here, this work can be expanded to explore optimisations of routing to perform different goals, such as in changing the utility function. A natural next step would be implementing these strategies in real-world SDN systems so that that could be tested and used on real-world networks. This would present new challenges, mainly centred around managing incomplete information and distributing control.

We did see success with the generalisation ability of GNNs which suggests that outside of the area of data-driven routing there may be other useful application to explore. Other aspects of network control have been investigated with ML and particularly RL techniques such as resource scheduling and rate controls which may benefit greatly from the application of GNN-based techniques.

