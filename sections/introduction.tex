\chapter{Introduction}

% this has to be done after the chapter start so can't put in top level :'(
\pagenumbering{arabic}
\setcounter{page}{1}

Routing has always been an integral part of the functioning of the internet as it is required for data to traverse a network from source to destination successfully. Historically, strategies have mainly focussed on those that can be calculated in a distributed manner, are functionally correct, and are robust to network changes all while providing reasonable performance (both in the time to calculate routes and their effect on the traffic itself). More recently, and especially with the advent of \ac{sdn}, there has been a concerted effort to exert more control over how traffic is routed. This control has either been used to implement policies that favour types of traffic, particular customers, or simply aim to achieve particular notions of performance on the network.

In the area of intradomain routing where this kind of control is possible, one particular focus has been on minimising link over-utilisation, as congestion can have significant impacts on network performance for the end-user. To minimise congestion successfully, such protocols must take into account how one traffic flow on the network impacts the other traffic flows, which leads us to data-driven routing. It is already known how to route optimally given a set of traffic demands over a network. However, in general, these demands cannot be known in advance. Therefore, there have been efforts to create routing schemes that will generally give good congestion performance no matter the particular traffic demands on the network (called oblivious routing). Further research has aimed to make these routing strategies include some notion of the current traffic.

A recent paper: "Learning To Route with Deep RL" examined whether we can use reinforcement learning to produce better routing strategies than the oblivious approach. It worked under the assumption that there is some regularity in sequences of traffic demands on a network and that using this regularity we can get closer to the optimal performance. This work produced some impressive results. However, once it has learnt to route on a particular network, due to the rigid structure of the neural network used to make the policy, this cannot be applied to a different network. This issue may seem relatively small, that is until one takes into account the fact that networks often change, often due to temporary outages. If, for example, a link or node in the network were to be temporarily off-line, then such a system would be unable to provide a routing scheme.

This project seeks to take the problem specified in ``Learning to Route with Deep RL'' as well as its solution and extend its domain to cover changing the structure of the network itself. In other words, the aim is to use \ac{rl} to enable the creation of close-to-optimal routings for a network, given a history of traffic demands on that network, and that this should be able to generalise both over different demand sequences for the same network and different demand sequences for entirely different networks. Graph generalisation is a useful goal as it allows a learned routing strategy to continue to work on networks suffering faults such as dropped links and also may mean that the model would not need to be retrained to be used on different networks. Finally, we aim to show that this is not just a toy problem and solution but leads to results on real-world datasets.

In summary, this study makes the following research contributions:
\begin{itemize}
  \item Provides an environment for experimenting with \ac{rl} in data-driven routing
  \item Designs a new mapping from edge weights to a fully specified multipath routing
  \item Introduces policy designs for approaching data-driven routing in a way that is generalisable to different network topologies
  \item Presents a comprehensive assessment of the performance of different techniques with a specific focus on generalisability over traffic patterns and network topologies.
\end{itemize}

The rest of the dissertation is structured as follows: chapter~\ref{chapter:background} introduces both previous research in this area and the research on which we base techniques in this work; chapter~\ref{chapter:problem} presents the formal specification of the problem, how we designed the environment, and techniques that made the problem and routing feasible to be performed using \ac{rl}; chapter~\ref{chapter:learning} describes how the \ac{rl} policy was designed and trained, and the structure of the \acp{gnn} used; chapter~\ref{chapter:evaluation} explains the evaluation framework and examines the results of the experiments performed; and chapter~\ref{chapter:conclusions} summaries the findings and contributions of the entire work as well as providing an insight into possible future work.
